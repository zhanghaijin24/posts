\documentclass[fontset=windows]{ctexart}

\usepackage{tikz}
\usepackage{nicematrix}
\usepackage{xcolor}
\usepackage{amsmath}
%\usepackage{xeCJK}
%\setCJKmainfont{微软雅黑}
\usetikzlibrary{graphs, positioning, quotes, shapes.geometric}

\author{张海金}
\title{动I-CRH380A-2645主机柜安装}

\begin{document}
\maketitle

\section{方法一}
\subsection{旧主板}
\begin{tikzpicture}[node distance=120pt]
  \node[draw,rounded corners] (旧板) {旧板};
  \node[draw,right=of 旧板] (系统盘) {系统盘};
  \node[draw,below=0.5 of 系统盘] (硬盘仓) {硬盘仓};
  \node[draw,below=0.5 of 硬盘仓] (显示屏电源) {显示屏电源};
  \node[draw,below=0.5 of 显示屏电源] (显示屏信号) {显示屏信号};
  \node[draw,below=0.5 of 显示屏信号] (开机键) {开机键};
  \node[draw,below=0.5 of 开机键] (电源) {电源};
  \node[draw,below=0.5 of 电源] (触摸屏信号) {触摸屏信号};

  \draw[->] (旧板) -- (系统盘);
  \draw[->] (旧板) -- (硬盘仓);
  \draw[->] (旧板) -- (显示屏电源);
  \draw[->] (旧板) -- (显示屏信号);
  \draw[->] (旧板) -- (开机键);
  \draw[->] (旧板) -- (电源);
  \draw[->] (旧板) -- (触摸屏信号);
   
\end{tikzpicture}

\subsection{新主板}
\begin{tikzpicture}[node distance=180pt]
  \node[draw,rounded corners] (新板) {新板};
  \node[draw,right=of 新板] (硬盘仓) {硬盘仓};
  \node[draw,below=0.5 of 硬盘仓] (显示屏电源) {显示屏电源};
  \node[draw,below=0.5 of 显示屏电源] (显示屏信号) {显示屏信号};
  \node[draw,below=0.5 of 显示屏信号] (开机键) {开机键};
  \node[draw,below=0.5 of 开机键] (电源) {电源};
  \node[draw,below=0.5 of 电源] (触摸屏信号) {触摸屏信号};
  \node[draw,below=0.5 of 触摸屏信号] (备用USB、网口、串口) {备用USB、
    网口、串口};

  \draw[->] (新板) -- (硬盘仓);
  \draw[->] (新板) -- (显示屏电源);
  \draw[->] (新板) -- (显示屏信号);
  \draw[->] (新板) -- (开机键);
  \draw[->] (新板) -- (电源);
  \draw[->] (新板) -- (触摸屏信号);
  \draw[->] (新板) -- (备用USB、网口、串口);
\end{tikzpicture}

\section{方法二}
\subsection{旧主板}
\begin{NiceTabular}{llc}[hvlines]
  \Block{9-1}{旧主板} & \Block{2-1}{1.硬盘仓} & 电源线4pin-2根\\
  & & 硬盘sata线-1根 \\
  & \Block{}{2.显示屏电源} & 1根\\
  & \Block{}{3.显示屏信号} & 1根\\
  & \Block{}{4.触摸屏信号} & 1根\\
  & \Block{}{5.开机键} & 一红一黑靠近电池\\
  & \Block{}{6.电源线4pin} & 6pin线\\
  & \Block{2-1}{7.系统盘} & 电源线4pin-1根\\
  & & 硬盘sata线-1根\\
  
\end{NiceTabular}

\subsection{新主板}
\begin{NiceTabular}{lll}[hvlines]
  M2 硬盘 & \Block{7-1}{新主板} & 1.硬盘仓\\
  内存条  & & 2.显示屏电源\\
  & & 3.显示屏信号\\
  & & 4.触摸屏信号\\
  & & 5.备用到USB、网口、串口\\
  & & 6.开机键\\
  & & 7.电源\\
  
\end{NiceTabular}
\end{document}